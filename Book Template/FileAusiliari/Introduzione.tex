		\titleformat{\chapter}
		[hang]
		{\Huge}
		{}
		{0em}
		{}
		[\Large {\begin{tikzpicture} [remember picture, overlay]
		\pgftext[right,x=14.75cm,y=0.2cm]{\HUGE\bfseries 
			Introduzione}
		\end{tikzpicture}}]
%%%%%%%%%%%%%%%%%%%%%%%%%%%%%%%%%%%%%%%%%%%%%%%%%%%%%%%%%%%%%%%%%%%%%%%%%%%%%%%%%
\chapter*{}\normalfont\addcontentsline{toc}{part}{Introduzione}


Lo scopo di questo documento è raccogliere appunti e note degli studenti del corso NLP, e creare un riassunto sostanzioso e fruibile a chiunque negli anni a venire, con la speranza che venga ampliato e modificato da futuri studenti.

"What is natural language and what do we want to do with it?", è la domanda di apertura del corso. Lo scopo del linguaggio è far comunicare due (o più) entità. Secondo Ferdinand De Saussure i parlanti si accordano per chiamare oggetti e concetti in determinate maniere. La ragion d'esistere delle parole non c'è, l'accezione è relativa ad un aspetto sociale. Parliamo infatti di linguistica *esterna*, ovvero studio del linguaggio come entità esterna all’individuo. Distinguiamo inoltre la *langue*, l'entità sociale propria di una comunità, canonizzata e dunque indrottinata ai cervelli, dalla *parole*, che è invece l'entità naturale, espressione con la propria cognizione (=attuazione individuale) della langue.
Chomsky tratta invece della linguistica interna, studio del linguaggio come capacità cognitiva dell’individuo, dove la distinzione è fra *competence*, capacità d apprendere, e *performance*, capacità di produzione del linguaggio.
I large language models lavorano sulla parole, anche se inizialmente l'NLP, le discipline Corpus Linguistics e gli EMNLP (empirical metaods for NLP), si concentravano sulla competence. Anche se la parole evolve continuamente, la langue stessa non è stabile: le parole nascono, muoiono e cambiano di significato. 